\yarlis\footnote{Yet Another Rogue-Like In Space} a pour objectif de s'inscrire dans la lignée de jeux tels \verb|Binding of Isaac| ou \verb|FTL| (les rogue-likes récents les plus connus). Il a également pour objectif de nous familiariser avec les mécaniques de création d'un jeu vidéo dans son ensemble, et à partir de zéro.

\section{Concept du jeu}

Dans \yarlis{}, le joueur incarne un chasseur de prime. Son objectif est d'éliminer des cibles (pirates de l'espace, communistes de l'espace ou que sais-je). Ces cibles sont cependant terrés dans leurs vaisseaux et c'est au joueur d'aller les débusquer.

Le fonctionnement d'une partie est le suivant :
\begin{itemize}
\item le joueur arrive à bord du vaisseau cible, au niveau des hangars;
\item sa cible est située dans la salle du capitaine, dans le poste de commandement;
\item les hangars et le poste de comandement sont situés à des étages différents;
\item le joueur doit donc passer par une succession d'étages;
\item les étages sont reliés par de ascenseurs;
\item les ascenceurs sont verrouillés jusqu'à ce que la condition de l'étage soit remplie (eg : tuer un boss);
\end{itemize}

Le jeu est un rogue-like classique en 2D. Les personnages du joueur sont persistés dans la victoire : si le joueur réussit son élimination, il peut conserver le personnage et ses équipements pour une partie suivante. La mort est cependant définitive.

\section{Bestiaire}

\subsection{Pirates de l'espace}


\section{Équipement}

\subsection{Armes}

