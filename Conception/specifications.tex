\documentclass[a4paper,12pt]{article}
\usepackage[utf8]{inputenc}
\usepackage[french]{babel}
\usepackage{fullpage}
\usepackage{graphicx}
\title{Spécifications du projet YARLIS\\- Yet Another Rogue-Like In Space -}
\author{Maxime PATTYN \& François-Xavier Matthews}

\begin{document}
\maketitle

\section{Principe}
YARLIS se veux être un projet permettant de comprendre le processus de création d'un jeu.
Le jeu sera donc un rogue-like, inspiré du jeu The Binding of Isaac, dans un univers futuriste, 
où le joueur contrôle un chasseur de prime au cours de différentes missions.\\
Dans ces missions, le joueur progresse dans des niveaux générés aléatoirement jusqu'à atteindre un boss
qui est sa cible.\\
Le jeu possèdera une composante RPG (sans quête) afin que le joueur puisse améliorer son personnage
(vie, energie, oxygène, caractéristiques des armes, etc).


\section{Maquette}

A compléter


\section{Caractéristiques du joueur}
Les joueur disposera des barres suivantes :
\begin{itemize}
  \item Vie ; 
  \item Energie (Munition des armes lourdes) ;
  \item Oxygène (Pour les pièces sans oxygène).
\end{itemize}


\section{Armes}

Les armes disponibles pour le joueur seront les suivantes.

\subsection{Armes Légères}
\begin{itemize}
  \item Blaster : arme tirant de petit traits d'énergie ;
  \item Couteau.
\end{itemize}

\subsection{Armes moyennes}
\begin{itemize}
  \item Shotgun ;
  \item Fusil-Blaster ;
  \item Fusil plasma ;
  \item Epée énergétique.
\end{itemize}

\subsection{Armes Lourdes}
\begin{itemize}
  \item Lance-Roquettes ;
  \item Minigun (Laser) ;
  \item Laser Spartan ;
  \item Lance-flamme.
\end{itemize}

\subsection{Améliorations}
Le joueur pourra prendre une arme de chaque section sur lui. Ces armes seront améliorables sur 
trois caractéristiques :
\begin{itemize}
  \item Vitesse de tir ;
  \item Puissance de tir ;
  \item Type de tir.
\end{itemize}

\section{Objets}

Objets de remplissement des barres :
\begin{itemize}
  \item Medikits ;
  \item Cellule énergétique ;
  \item Recharge d'oxygène.
\end{itemize}

Autre objets :
\begin{itemize}
  \item Objets équivalents des clés : cartes magnétiques / kits de piratage ? ;
  \item Cellule d'amélioration ;
  \item etc.
\end{itemize}

Ces objets seront ramassés dans des caisses ou des coffres. Ces coffre seront à ouvrir avec une clé
ou a casser avec une arme lourde explosive. Les ennemis peuvent aussi possiblement en lacher.

\section{Ennemis}
Les ennemis seront déterminés par les caractéristiques suivantes :
\begin{itemize}
  \item Vie ;
  \item Vitesse / Puissance / Type d'attaque ;
  \item Déplacements (vitesse / type) ;
  \item Drops ;
  \item IA (+- intelligente).
\end{itemize}

Les boss sont pour l'instant un ennemi commun plus puissant.

\section{Environnement}
Idées pour les environnements :
\begin{itemize}
  \item vitre pouvant être cassés, vidant l'oxygène de la pièce ;
  \item pièges ;
  \item pièces types (salles, couloirs, etc).
\end{itemize}

\end{document}
